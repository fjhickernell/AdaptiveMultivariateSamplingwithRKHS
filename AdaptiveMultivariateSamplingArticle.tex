\documentclass[]{mcom-l}


\usepackage{latexsym,amsfonts,amsmath,amssymb,amsthm,epsfig,extdash,multirow}
\usepackage{stackrel,tabularx,mathtools,enumitem,longtable,xspace}
\usepackage[dvipsnames]{xcolor}
%\usepackage[numbers,sort&compress]{natbib}
\usepackage{hyperref,accents, booktabs}
\usepackage{algorithm, algorithmicx}
\usepackage{anyfontsize}
\usepackage{cleveref}
\usepackage{wrapfig}
\usepackage[font=small,labelfont=bf]{caption}

\usepackage{algpseudocode}
\usepackage{algorithm, algorithmicx}
\algnewcommand\algorithmicparam{\textbf{Parameters:}}
\algnewcommand\PARAM{\item[\algorithmicparam]}
\algnewcommand\algorithmicinput{\textbf{Input:}}
\algnewcommand\INPUT{\item[\algorithmicinput]}
\algnewcommand\RETURN{\State \textbf{Return }}
\usepackage{algpseudocode}

\theoremstyle{remark}
\newtheorem{example}{Example}
%%list of acronyms with links
\newcommand{\QMCSoft}{QMCSoft\xspace}
\newcommand{\GAIL}{GAIL\xspace}
\newcommand{\QMC}{QMC\xspace}
\newcommand{\IIDMC}{IID MC\xspace}
\newcommand{\SAMSIQMC}{SAMSI-QMC\xspace}
\newcommand{\SciPy}{SciPy\xspace}
\newcommand{\GSL}{GSL\xspace}
\newcommand{\NAG}{NAG\xspace}
\newcommand{\MATLAB}{MATLAB\xspace}
\newcommand{\Chebfun}{Chebfun\xspace}
\newcommand{\Rlang}{R\xspace}
\newcommand{\Julia}{Julia\xspace}


\textwidth6.5in
\setlength{\oddsidemargin}{0in}
\setlength{\evensidemargin}{0in}
\textheight9.0in
%\textheight9.1in

\newtheorem{theorem}{theorem}


\providecommand{\FJHickernell}{Hickernell}
\newcommand{\hf}{\widehat{f}}
\newcommand{\hg}{\widehat{g}}
\newcommand{\hI}{\hat{I}}
\newcommand{\hatf}{\hat{f}}
\newcommand{\hatg}{\hat{g}}
\newcommand{\tf}{\widetilde{f}}
\newcommand{\tbf}{\tilde{\bff}}
%\DeclareMathOperator{\Pr}{\mathbb{P}}

% Math operators
\DeclareMathOperator{\cost}{COST}
\DeclareMathOperator{\comp}{COMP}
\DeclareMathOperator{\loss}{loss}
\DeclareMathOperator{\lof}{lof}
\DeclareMathOperator{\reg}{reg}
\DeclareMathOperator{\CV}{CV}
\DeclareMathOperator{\size}{wd}
\DeclareMathOperator{\GP}{\mathcal{G} \! \mathcal{P}}
\DeclareMathOperator{\erf}{erf}
\DeclareMathOperator*{\argmax}{arg\,max}
\DeclareMathOperator*{\argmin}{arg\,min}
\DeclareMathOperator{\QOI}{QOI} %Quantity of Interest
\DeclareMathOperator{\POI}{POI} %Parameter of Interest
\DeclareMathOperator{\Ans}{ANS}
\DeclareMathOperator{\Var}{Var}
%\DeclareMathOperator{\APP}{\widehat{\QOI}}
\DeclareMathOperator{\SURR}{SM} %surrogate model
\DeclareMathOperator{\STREND}{ST} %surrogate trend
\DeclareMathOperator{\SVAR}{SV} %surrogate variation
\DeclareMathOperator{\SVARERR}{SVU} %surrogate variation uncertainty
\newcommand{\MLS}{\textrm{MLS}\xspace} %distance weighted least squares, also known as moving least squares
\DeclareMathOperator{\ALG}{ALG}
\DeclareMathOperator{\ERR}{ERR}
\DeclareMathOperator{\VAL}{ACQ}
\DeclareMathOperator{\OPER}{OPER}
\DeclareMathOperator{\INT}{INT}
\DeclareMathOperator{\MIN}{MIN}
\DeclareMathOperator{\ID}{ID}
\DeclareMathOperator{\APPMIN}{\widehat{\MIN}}
\DeclareMathOperator{\APPID}{\widehat{\ID}}
\DeclareMathOperator{\MINVAL}{MINACQ}
\DeclareMathOperator{\IDVAL}{IDACQ}
\DeclareMathOperator{\SURRERR}{SU}
\DeclareMathOperator{\MINERR}{MERR}
\DeclareMathOperator{\IDERR}{IDERR}
\DeclareMathOperator{\Prob}{\mathbb{P}}
\DeclareMathOperator{\diag}{diag}
\DeclareMathOperator{\dist}{dist}
\DeclareMathOperator{\filldis}{fill}
\DeclareMathOperator{\sep}{sep}
\DeclareMathOperator{\avg}{avg}
\DeclareMathOperator{\vol}{vol}
\DeclareMathOperator{\cov}{cov}
\DeclareMathOperator{\APP}{APP}
\newcommand{\TREND}{\textup{T}}
\newcommand{\VAR}{\textup{V}}
\newcommand{\LS}{\textup{LS}}


\newcommand{\reals}{{\mathbb{R}}}
\newcommand{\naturals}{{\mathbb{N}}}
\newcommand{\natzero}{{\mathbb{N}_0}}
\newcommand{\integers}{{\mathbb{Z}}}
\def\expect{{\mathbb{E}}}
\def\il{\left \langle}
\def\ir{\right \rangle}
\def\e{\varepsilon}
\def\g{\gamma}
\def\l{\lambda}
\def\b{\beta}
\def\a{\alpha}
\def\lall{\Lambda^{{\rm all}}}
\def\lstd{\Lambda^{{\rm std}}}

\newcommand{\vf}{\boldsymbol{f}}
\newcommand{\hV}{\widehat{V}}
\newcommand{\tV}{\widetilde{V}}
\newcommand{\fraku}{\mathfrak{u}}
\newcommand{\hcut}{\mathfrak{h}}
\newcommand{\tOmega}{\widetilde{\Omega}}
\newcommand{\tvarrho}{\widetilde{\varrho}}

\newcommand{\bbE}{\mathbb{E}}
\newcommand{\tQ}{\widetilde{Q}}
\newcommand{\mA}{\mathsf{A}}
\newcommand{\mB}{\mathsf{B}}
\newcommand{\mC}{\mathsf{C}}
\newcommand{\mD}{\mathsf{D}}
\newcommand{\mG}{\mathsf{G}}
\newcommand{\mH}{\mathsf{H}}
\newcommand{\mI}{\mathsf{I}}
\newcommand{\bbK}{\mathbb{K}}
\newcommand{\mK}{\mathsf{K}}
\newcommand{\tmK}{\widetilde{\mathsf{K}}}
\newcommand{\mL}{\mathsf{L}}
\newcommand{\mM}{\mathsf{M}}
\newcommand{\mP}{\mathsf{P}}
\newcommand{\mQ}{\mathsf{Q}}
\newcommand{\mU}{\mathsf{U}}
\newcommand{\mR}{\mathsf{R}}
\newcommand{\mX}{\mathsf{X}}
\newcommand{\mPhi}{\mathsf{\Phi}}
\newcommand{\mPsi}{\mathsf{\Psi}}
\newcommand{\mLambda}{\mathsf{\Lambda}}
\newcommand{\cube}{[0,1]^d}
\newcommand{\design}{\{\bx_i\}_{i=1}^n}




\newcommand{\bone}{\boldsymbol{1}}
\newcommand{\bzero}{\boldsymbol{0}}
\newcommand{\binf}{\boldsymbol{\infty}}
\newcommand{\ba}{{\boldsymbol{a}}}
\newcommand{\bb}{{\boldsymbol{b}}}
\newcommand{\bc}{{\boldsymbol{c}}}
\newcommand{\bd}{{\boldsymbol{d}}}
\newcommand{\be}{{\boldsymbol{e}}}
\newcommand{\bff}{{\boldsymbol{f}}}
\newcommand{\bhh}{{\boldsymbol{h}}}
\newcommand{\beps}{{\boldsymbol{\varepsilon}}}
\newcommand{\tbeps}{\tilde{\beps}}
\newcommand{\bx}{{\boldsymbol{x}}}
\newcommand{\bX}{{\boldsymbol{X}}}
\newcommand{\bh}{{\boldsymbol{h}}}
\newcommand{\bj}{{\boldsymbol{j}}}
\newcommand{\bk}{{\boldsymbol{k}}}
\newcommand{\vell}{{\boldsymbol{\ell}}}
\newcommand{\bL}{{\boldsymbol{L}}}
\newcommand{\bg}{{\boldsymbol{g}}}
\newcommand{\bn}{{\boldsymbol{n}}}
\newcommand{\br}{{\boldsymbol{r}}}
\newcommand{\bv}{{\boldsymbol{v}}}
\newcommand{\bu}{{\boldsymbol{u}}}
\newcommand{\by}{{\boldsymbol{y}}}
\newcommand{\bt}{{\boldsymbol{t}}}
\newcommand{\bU}{{\boldsymbol{U}}}
\newcommand{\bz}{{\boldsymbol{z}}}
\newcommand{\bvarphi}{{\boldsymbol{\varphi}}}
\newcommand{\bgamma}{{\boldsymbol{\gamma}}}
\newcommand{\bphi}{{\boldsymbol{\phi}}}
\newcommand{\bpsi}{{\boldsymbol{\psi}}}
\newcommand{\btheta}{{\boldsymbol{\theta}}}
\newcommand{\bnu}{{\boldsymbol{\nu}}}
\newcommand{\balpha}{{\boldsymbol{\alpha}}}
\newcommand{\bbeta}{{\boldsymbol{\beta}}}
\newcommand{\bo}{{\boldsymbol{\omega}}}  %GF added
\newcommand{\newton}[2]{\left(\begin{array}{c} #1\\ #2\end{array}\right)}
\newcommand{\anor}[2]{\| #1\|_{\mu_{#2}}}
\newcommand{\satop}[2]{\stackrel{\scriptstyle{#1}}{\scriptstyle{#2}}}
\newcommand{\setu}{{\mathfrak{u}}}

\newcommand{\me}{\textup{e}}
\newcommand{\mi}{\textup{i}}
\def\d{\textup{d}}
\def\dif{\textup{d}}
\newcommand{\cc}{\mathcal{C}}
\newcommand{\cb}{\mathcal{B}}
\newcommand{\cl}{L}
\newcommand{\ct}{\mathfrak{T}}
\newcommand{\cx}{{\Omega}}
\newcommand{\cala}{{\mathcal{A}}}
\newcommand{\calc}{{\mathcal{C}}}
\newcommand{\calf}{{\mathcal{F}}}
\newcommand{\calfd}{{\calf_d}}
\newcommand{\calh}{{\mathcal{H}}}
\newcommand{\tcalh}{{\widetilde{\calh}}}
\newcommand{\calI}{{\mathcal{I}}}
\newcommand{\calhk}{\calh_d(K)}
\newcommand{\calg}{{\mathcal{G}}}
\newcommand{\calgd}{{\calg_d}}
\newcommand{\caln}{{\mathcal{N}}}
\newcommand{\calp}{{\mathcal{P}}}
\newcommand{\cals}{{\mathcal{S}}}
\newcommand{\cL}{\mathcal{L}}
\newcommand{\cP}{\mathcal{P}}
\newcommand{\cT}{\mathcal{T}}
\newcommand{\cK}{\mathcal{K}}
\newcommand{\fA}{\mathfrak{A}}
\newcommand{\fC}{\mathfrak{C}}
\newcommand{\fF}{\mathfrak{F}}
\newcommand{\fL}{\mathfrak{L}}
\newcommand{\fU}{\mathfrak{U}}
\newcommand{\hS}{\widehat{S}}

\def\abs#1{\ensuremath{\left \lvert #1 \right \rvert}}
\newcommand{\bigabs}[1]{\ensuremath{\bigl \lvert #1 \bigr \rvert}}
\newcommand{\norm}[2][{}]{\ensuremath{\left \lVert #2 \right \rVert}_{#1}}
\newcommand{\ip}[3][{}]{\ensuremath{\left \langle #2, #3 \right \rangle_{#1}}}
\newcommand{\bignorm}[2][{}]{\ensuremath{\bigl \lVert #2 \bigr \rVert}_{#1}}
\newcommand{\Bignorm}[2][{}]{\ensuremath{\Bigl \lVert #2 \Bigr \rVert}_{#1}}
\newcommand{\calm}{{\mathfrak{M}}}

\newcommand{\des}{\{\bx_i\}}
\newcommand{\desinf}{\{\bx_i\}_{i=1}^{\infty}}
\newcommand{\desn}{\{\bx_i\}_{i=1}^n}
\newcommand{\wts}{\{g_i\}_{i=1}^N}
\newcommand{\wtsn}{\{g_i\}_{i=1}^N}
\newcommand{\datan}{\{y_i\}_{i=1}^N}

%FJH added
\newcommand{\Order}{\mathcal{O}}
\newcommand{\ch}{\mathcal{H}}
\newcommand{\tch}{{\widetilde{\ch}}}
\newcommand{\veps}{\boldsymbol{\varepsilon}}
\DeclareMathOperator{\best}{best}
\newcommand{\hmu}{\hat{\mu}}
\newcommand{\hsigma}{\hat{\sigma}}
\newcommand{\tK}{\widetilde{K}}
%\newcommand{\Matlab}{{\sc Matlab}\xspace}
\newcommand{\abstol}{\varepsilon_{\text{a}}}
\newcommand{\reltol}{\varepsilon_{\text{r}}}

\newcommand\starred[1]{\accentset{\star}{#1}}

\newcommand{\designInf}{\{\bx_i\}_{i=1}^\infty}
\newcommand{\dataN}{\bigl\{\bigl(\bx_i,f(\bx_i)\bigr)\bigr\}_{i=1}^n}
\newcommand{\dataNp}{\bigl\{\bigl(\bx_i,f(\bx_i)\bigr)\bigr\}_{i=1}^{n'}}
\newcommand{\dataNo}{\bigl\{\bigl(\bx_i,f(\bx_i)\bigr)\bigr\}_{i=1}^{n_0}}
\newcommand{\ErrN}{\ERR\bigl(\dataN,n\bigr)}
\newcommand{\fint}{f_{\text{int}}}
\newcommand{\inflate}{\fC}

\newcommand{\FredNote}[1]{{\color{blue}Fred: #1}}

\title{Adaptive Multivariate Sampling to Accelerate Discovery}
\author{}
\date{\today}

\begin{document}

\maketitle

\section{Introduction}
Rigorous error bounds for numerical approximations typically consist of the norm of the input function multiplied by the norm of the error operator.  Because the norm of the input function is unknown, these error bounds cannot tell us whether a prescribed error tolerance has been met.  Adaptive numerical algorithms typically use heuristic data-driven error bounds, but then we don't know under what conditions they are trustworthy.

We propose an adaptive function approximation algorithm based on rigorous, data-driven error bounds.  This algorithm assumes that the function to be approximated lies in a \emph{candidate set}, which takes the shape of a cone, $\cc$.  The definition of $\cc$ formalizes the idea that \emph{what you observe is nearly what you get}.  If $f \in \cc$, then any constant multiple of $f$ is also in $\cc$.  The underlying numerical approximation is the well-known reproducing kernel Hilbert space (RKHS) minimum norm interpolant, and its error bound can be written explicitly.  Baking into $\cc$ the assumption that the norm of a function is not much worse than the norm of its interpolant allows us to construct a rigorous adaptive function approximation algorithm. Given a black-box input function, $f$, and an error tolerance, $\varepsilon$, our algorithm returns an approximation, $\ALG(f,\varepsilon) \in L^\infty$---defined only in terms of function values---for which 
\begin{equation} \label{eq:errorcrit}
\norm[\infty]{f - \ALG(f,\varepsilon)} \le \varepsilon \qquad \forall f\in \cc.
\end{equation}

There are several forms that adaption can take, including
\begin{itemize}
    \item Adaptive choice of the number of function data needed, $n$,
    \item Adaptive choice of the data sites, $\{\bx_1, \bx_2, \ldots\}$,
    \item Adaptive choice of the function space, $\calf$, for which the input function is typical.
\end{itemize}
We construct an algorithm utilizing the first two kinds of adaption in Section \ref{sec:fixedF} and incorporate the third kind of adaption in Section \ref{sec:adaptF}.


Any algorithm that claims to satisfy \eqref{eq:errorcrit} cannot possibly do so for a candidate set corresponding to an infinite dimensional vector space.  If there were such an algorithm, then $\ALG(0,\varepsilon)$ would terminate after evaluating the zero function at some $\bx_1, \ldots, \bx_n$.  Since $\cc$ is an infinite dimensional vector space, there exists some $g \in \cc$, for which $g(\bx_1) = \cdots = g(\bx_n) = 0$ but for which $\norm[\infty]{g} > 2 \varepsilon$.  That is, $g$ looks like $0$ to the algorithm, but is not close enough to $0$.  In this case, $\ALG(g,\varepsilon) = \ALG(0,\varepsilon)$, and it is impossible for the error criterion \eqref{eq:errorcrit} to be satisfied for both $f=0$ and $f = g$.

Function approximation algorithms are often constructed to satisfy \eqref{eq:errorcrit} for a ball candidate set, e.g., $\cc = \{f \in \calf : \norm[\calf]{f} \le 1\}$.  Choosing this convex, symmetric  candidate set guarantees that adaption is of no use \cite{Bak71}.  Here we choose candidate sets that are unbounded and non-convex, which are amenable to adaptive algorithms.

The algorithms constructed here put a premium on the cost of obtaining function values.  We have in mind the situation where one function value may be the result of an expensive simulation costing hours or days of CPU time, denoted $\$(f)$.  Our algorithm produces an inexpensive surrogate or approximation of the expensive simulation.  If $n$ function values are required to obtain a satisfactory function approximation, then we do not mind if the manipulation of that data to form our function approximation requires $\Order(n^p)$ operations, since practically speaking this will be smaller than $\$(f) n$.  RKHS minimum  norm interpolants typically require $\Order(n^3)$ operations, and we may require a bit more to obtain a satisfactory error bound.


\FredNote{more} \cite{Hic97a}


%%%%%%%%%%%%%%%%%%%%%%%%%%%%%%%%%%%%%%%%%%%%%%%%%%%%%%%%%%%%%%%%%%%%%%
%%%%%%%%%%%%%%%%%%%%%%%%%%%%%%%%%%%%%%%%%%%%%%%%%%%%%%%%%%%%%%%%%%%%%%
\section{Adaptive Function Approximation for a Fixed Hilbert Space} \label{sec:fixedF}
%%%%%%%%%%%%%%%%%%%%%%%%%%%%%%%%%%%%%%%%%%%%%%%%%%%%%%%%%%%%%%%%%%%%%%
%%%%%%%%%%%%%%%%%%%%%%%%%%%%%%%%%%%%%%%%%%%%%%%%%%%%%%%%%%%%%%%%%%%%%%

%%%%%%%%%%%%%%%%%%%%%%%%%%%%%%%%%%%%%%%%%%%%%%%%%%%%%%%%%%%%%%%%%%%%%%
%%%%%%%%%%%%%%%%%%%%%%%%%%%%%%%%%%%%%%%%%%%%%%%%%%%%%%%%%%%%%%%%%%%%%%
\subsection{The Basic Idea}
%%%%%%%%%%%%%%%%%%%%%%%%%%%%%%%%%%%%%%%%%%%%%%%%%%%%%%%%%%%%%%%%%%%%%%
%%%%%%%%%%%%%%%%%%%%%%%%%%%%%%%%%%%%%%%%%%%%%%%%%%%%%%%%%%%%%%%%%%%%%%
Suppose that $f$, belongs to an RKHS $\calf$ of functions with domain $\cx$.  Let $K: \Omega \times \Omega \to \reals$ denote the reproducing kernel.  Let $\mX = (\bx_1, \ldots, \bx_n)^T \in \cx^n \subseteq \reals^{n \times d}$ be an array of $n$ data sites, and let $\by  = f(\mX) \in \reals^n$ denote the array of function values at these data sites.  
The minimum norm interpolant of $f$ is 
\begin{subequations} \label{eq:RKHSAPP}
\begin{align} 
\APP(\mX,\by) &= \sum_{i=1}^n c_i K(\bx_i,\cdot) = \bc^T K(\mX,\cdot) =  K(\cdot, \mX) \mK^{-1} \by \, \\
 \text{where } & \qquad \bc = \mK^{-1} \by, \quad \mK = K(\mX,\mX) = \bigl( K(\bx_i,\bx_j) \bigr)_{i,j=1}^n,  \\
& \qquad  K(\mX,\bx) = \bigl(K(\bx,\bx_i) \bigr)_{i=1}^n =  K(\bx, \mX)^T.
\end{align}
\end{subequations}
This has a known pointwise error bound of
\begin{align}
\label{eq:RKHSErrBd}
\abs{f(\bx) - \APP(\mX,\by)(\bx)} & \le \sqrt{K(\bx,\bx) - K(\bx,\mX) \mK^{-1} K(\mX,\bx)} \, \bignorm[\calf]{f - \APP(f)} \\
\nonumber
& \le \sqrt{K(\bx,\bx) - K(\bx,\mX) \mK^{-1} K(\mX,\bx) } \, \norm[\calf]{f} .
\end{align}

Data-based approximation $\APP(\mX,\by)$ in \eqref{eq:RKHSAPP} is an ingredient in our adaptive algorithm, but error bound \eqref{eq:RKHSErrBd} contains the unknown factor $\norm[\calf]{f - \APP(\mX,\by)}$ does cannot be computed from function data.  What we can compute from data is $\bignorm[\calf]{\APP(f)} = \sqrt{\by^T \mK^{-1} \by}$.  Moreover, it is known that the error is orthogonal to the approximation, so by the Pythagorean Theorem,  $\norm[\calf]{f}^2  = \bignorm[\calf]{\APP(f)}^2 + \bignorm[\calf]{f - \APP(f)}^2$.  To establish a data-driven error bound, we consider a candidate set
\begin{align} \label{eq:RKHScone}
\calc &:= \Bigl\{f \in \calf : \norm[\calf]{f}^2 \le (1 + A^2(\mX)) \bignorm[\calf]{\APP(\mX,\by)}^2 \quad \forall \mX \in \Omega^d, \ \by = f(\mX) \Bigr \} \\
\nonumber
& = \Bigl \{f \in \calf : \bignorm[\calf]{f - \APP(f)} \le A(\mX) \bignorm[\calf]{\APP(\mX,\by)} \quad \forall \mX \in \Omega^d, \ \by = f(\mX) \Bigr \},
\end{align}
where $A(\mX)$ is positive and fixed in advance. Larger $A(\mX)$ yields a more inclusive candidate set.  This candidate set is a cone because if $f \in \calc$, then $c f \in \calc$ for any real $c$. 

The intuition leading to the definition of our cone candidate set is \emph{what you have not seen is not much worse than what you can see}. All adaptive algorithms are based on this philosophy. The definition of the cone in \eqref{eq:RKHScone} is a way of formalizing this key idea. 

Because $\bignorm[\calf]{\APP(\mX,\by)} = \sqrt{\by^T \mK^{-1} \by}$, error bound \eqref{eq:RKHSErrBd} implies an error bound that can be computed solely based on the output data: 
\begin{subequations} \label{eq:DataErrBd}
\begin{align}
    \abs{f(\bx) - \APP(\mX,\by)(\bx)} & \le   A(\mX) \sqrt{[K(\bx,\bx) - K(\bx,\mX) \mK^{-1} K(\mX,\bx) ] \, [\by^T \mK^{-1} \by] } \qquad \forall f \in \calc, \\
    \norm[\infty]{f - \APP(\mX,\by)} & \le   A(\mX) \sqrt{\norm[\infty]{K(\cdot,\cdot) - K(\cdot,\mX) \mK^{-1} K(\mX,\cdot)} \, [\by^T \mK^{-1} \by] } \qquad \forall f \in \calc.
\end{align}
\end{subequations}
This data-based error bound provides what is needed for our adaptive algorithm.  Note that $n$ is implicit in the definition of $\mX$, but we do not show this explicitly to for notational simplicity.

\begin{algorithm}[H]
\caption{Adaptive Sample Size $\ALG$ for Multivariate Function Approximation \label{alg:basicadapt}}
	\begin{algorithmic}
	\PARAM the RKHS space $\calf$, a sequence of data sites $\{\bx_1, \bx_2, \ldots \}$, an initial sample size $n_0$, the factor $A(\cdot)$
	\INPUT a black-box function, $f \in \cc$; an absolute error tolerance, $\varepsilon>0$

    \Ensure Error criterion $\norm[\infty]{f - \ALG(f,\varepsilon)} \le \varepsilon$

   \State Let $n \leftarrow n_0 -1$

\Repeat

\State Let $n \leftarrow n + 1$

\State Compute $\by$, $A(\mX)$, $K(\mX,\mX)$, and $K(\mX,\cdot)$

\Until $A(\mX) \sqrt{\norm[\infty]{K(\cdot,\cdot) - K(\cdot,\mX) \mK^{-1} K(\mX,\cdot)} \, [\by^T \mK^{-1} \by] }  \le \varepsilon$

\RETURN $\ALG(f,\varepsilon) = \APP(\mX,\by)$

\end{algorithmic}
\end{algorithm}

This algorithm adaptively chooses the sample size so that the error criterion must be satisfied for all functions in the candidate set $\cc$.  For a well chosen design, $\norm[\infty]{K(\cdot,\cdot) - K(\cdot,\mX) \mK^{-1} K(\mX,\cdot)}$ tends to zero as $n$ increases.  A larger function yields larger function data $\by$, which implies that a larger sample size will be required to achieve the error criterion.

However, there are several deficiencies in this algorithm.  The choice of the design does not yet depend in on the function data.  There is no assurance that the RKHS specified by the choice of $K$ fits $f$.  There is also the matter of how to choose $A(\mX)$.


%%%%%%%%%%%%%%%%%%%%%%%%%%%%%%%%%%%%%%%%%%%%%%%%%%%%%%%%%%%%%%%%%%%%%%
%%%%%%%%%%%%%%%%%%%%%%%%%%%%%%%%%%%%%%%%%%%%%%%%%%%%%%%%%%%%%%%%%%%%%%
\subsection{Choosing a Reasonable Factor $A$?}
%%%%%%%%%%%%%%%%%%%%%%%%%%%%%%%%%%%%%%%%%%%%%%%%%%%%%%%%%%%%%%%%%%%%%%
%%%%%%%%%%%%%%%%%%%%%%%%%%%%%%%%%%%%%%%%%%%%%%%%%%%%%%%%%%%%%%%%%%%%%%

The candidate set depends on the factor $A(\mX)$, which, according to \ref{eq:RKHScone}, reflects how much larger one expects $\bignorm[\calf]{f  - \APP(\mX,\by)}$ might be than $\bignorm[\calf]{\APP(\mX,\by)}$ is.  It is natural for $A(\mX)$ to depend  on the quality of the design and the number of data sites.  One measure of the quality of the design is 
\begin{equation*}
B(\mX) : = \frac{ \sup \{\norm[\infty]{f} :  \norm[\calf]{f}  \le 1 , \ f(\mX) = 0 \}   }  {\sup \{ \norm[\infty]{f}  : \norm[\calf]{f}  \le 1  \} }
= \sqrt{ \frac{\norm[\infty]{K(\cdot,\cdot) - K(\cdot,\mX) \mK^{-1} K(\mX,\cdot)}}{\norm[\infty]{K(\cdot,\cdot)}}} \le 1.
\end{equation*}
As the sample size increases using well-placed data, $B(\mX)$ tends to zero.

The choice of $A(\mX)$ is rather a value judgment.  We propose
\begin{equation} \label{eq:an}
A(\mX): = \begin{cases} \displaystyle
\frac{A_\infty B_0}{B_0 - B(\mX)}, & B(\mX) < B_0, \\
\infty, & B(\mX) \ge B_0.
\end{cases}
\end{equation}
The parameter $A_\infty > 0$ represents the limiting value of $A(\mX)$ as the design tends to perfection.  The parameter $B_0$ represents the maximum value of $B(\mX)$ for which one is willing to infer a bound on $\bignorm[\calf]{f  - \APP(\mX,\by)}$ in terms of $\bignorm[\calf]{\APP(\mX,\by)}$.  Making $A_\infty$ larger or $B_0$ smaller makes the candidate set, $\cc$ more inclusive.

\begin{example}
	
	
	
\end{example}


Here we use one 1-d example to illustrate it.

 A commonly used reproducing kernel is the following member of the Mat\'ern family:
\[
K(\bt,\bx) = (1 + \theta \norm[2]{\bt-\bx}) \exp(-\theta\norm[2]{\bt-\bx}),
\]
where $\theta = 1.$
We have \[f: x \mapsto \exp(-6x) \sin(8x+0.1) - 0.1.\]
Consider the $n=10$ point design $\mX = (0, 0.1, \ldots, 0.6, 0.8, 0.9, 1)^T$

%%%%%%%%%%%%%%%%%%%%%%%%%%%%%%%%%%%%%%%%%%%%%%%%%%%%%%%%%%%%%%%%%%%%%%
%%%%%%%%%%%%%%%%%%%%%%%%%%%%%%%%%%%%%%%%%%%%%%%%%%%%%%%%%%%%%%%%%%%%%%
\subsection{Computational Cost}
%%%%%%%%%%%%%%%%%%%%%%%%%%%%%%%%%%%%%%%%%%%%%%%%%%%%%%%%%%%%%%%%%%%%%%
%%%%%%%%%%%%%%%%%%%%%%%%%%%%%%%%%%%%%%%%%%%%%%%%%%%%%%%%%%%%%%%%%%%%%%
Although we explained in the introduction that we are primarily concerned with the computational cost of obtaining function data, we will count the cost of our algorithms.  Explicitly showing the $n$ dependence, the sequence of data vectors, $\by_1, \ldots, \by_n$, cost $\$(f)n$, explicitly showing that the cost of a function value might be quite large.  Moreover, $K(\bx,\mX_1), \ldots, K(\bx,\mX_n)$ can  be be computed iteratively in $\Order(n)$ operations.  The Cholesky decompositions of $\mK_1, \ldots, \mK_n$ and $\mK_1^{-1}, \ldots, \mK_n^{-1}$ can all be computed in $\Order(n^3)$ operations as shown in the appendix.  If a reasonable approximation of $\norm[\infty]{K(\cdot,\cdot) - K(\cdot,\mX) \mK^{-1} K(\mX,\cdot)}$ can be made using $\Order(N)$ values of  $K(\cdot,\cdot) - K(\cdot,\mX) \mK^{-1} K(\mX,\cdot)$, then the computational cost of Algorithm \ref{alg:basicadapt} is
\begin{equation}
\Order\bigl(\$(f)n + n^3 + Nn^2 \bigr).
\end{equation}



%%%%%%%%%%%%%%%%%%%%%%%%%%%%%%%%%%%%%%%%%%%%%%%%%%%%%%%%%%%%%%%%%%%%%%
%%%%%%%%%%%%%%%%%%%%%%%%%%%%%%%%%%%%%%%%%%%%%%%%%%%%%%%%%%%%%%%%%%%%%%
\subsection{Adaptive Sampling }
%%%%%%%%%%%%%%%%%%%%%%%%%%%%%%%%%%%%%%%%%%%%%%%%%%%%%%%%%%%%%%%%%%%%%%
%%%%%%%%%%%%%%%%%%%%%%%%%%%%%%%%%%%%%%%%%%%%%%%%%%%%%%%%%%%%%%%%%%%%%%

\begin{equation} \label{eq:nextsample}
\bx_{n+1} = \argmax_{\bx \in \Omega}  A_n \sqrt{[K(\bx,\bx) - \bk^T(\bx) \mK^{-1} \bk(\bx)] \, [\by^T \mK^{-1} \by] }.
\end{equation}




%%%%%%%%%%%%%%%%%%%%%%%%%%%%%%%%%%%%%%%%%%%%%%%%%%%%%%%%%%%%%%%%%%%%%%%%%%
%%%%%%%%%%%%%%%%%%%%%%%%%%%%%%%%%%%%%%%%%%%%%%%%%%%%%%%%%%%%%%%%%%%%%%%%%%
\section{Function Approximation when the Hilbert Space Is Inferred} \label{sec:adaptF}
%%%%%%%%%%%%%%%%%%%%%%%%%%%%%%%%%%%%%%%%%%%%%%%%%%%%%%%%%%%%%%%%%%%%%%%%%%
%%%%%%%%%%%%%%%%%%%%%%%%%%%%%%%%%%%%%%%%%%%%%%%%%%%%%%%%%%%%%%%%%%%%%%%%%%



%%%%%%%%%%%%%%%%%%%%%%%%%%%%%%%%%%%%%%%%%%%%%%%%%%%%%%%%%%%%%%%%%%%%%%
%%%%%%%%%%%%%%%%%%%%%%%%%%%%%%%%%%%%%%%%%%%%%%%%%%%%%%%%%%%%%%%%%%%%%%
\subsection{ Infer the value of $\theta$}
%%%%%%%%%%%%%%%%%%%%%%%%%%%%%%%%%%%%%%%%%%%%%%%%%%%%%%%%%%%%%%%%%%%%%%
%%%%%%%%%%%%%%%%%%%%%%%%%%%%%%%%%%%%%%%%%%%%%%%%%%%%%%%%%%%%%%%%%%%%%%

One improvement is to infer the value of $\theta$ inherent in the definition of $K$ in rather than to set it arbitrarily. The empirical Bayes perspective leads to the following choice: 
\begin{equation} \label{eq:thetEB}
\theta_{\textup{EB}} = \argmin_\theta \left[\frac 1n \log \bigl( \det(\mK_\theta) \bigr) + \log \bigl ( \by^T \mK_\theta^{-1} \by \bigr)\right].
\end{equation}



\bibliographystyle{amsplain}
\bibliography{FJHown23,FJH23}

%%%%%%%%%%%%%%%%%%%%%%%%%%%%%%%%%%%%%%%%%%%%%%%%%%%%%%%%%%%%%%%%%%%%%%
%%%%%%%%%%%%%%%%%%%%%%%%%%%%%%%%%%%%%%%%%%%%%%%%%%%%%%%%%%%%%%%%%%%%%%
\section*{Appendix}
%%%%%%%%%%%%%%%%%%%%%%%%%%%%%%%%%%%%%%%%%%%%%%%%%%%%%%%%%%%%%%%%%%%%%%
%%%%%%%%%%%%%%%%%%%%%%%%%%%%%%%%%%%%%%%%%%%%%%%%%%%%%%%%%%%%%%%%%%%%%%
The Gram matrix, $\mK$ defined in \eqref{eq:RKHSAPP} is positive definite, and therefore has a Cholesky decomposition, $\mK = \mL \mD \mL^T$, where $\mL$ is a lower unitriangular matrix and $\mD$ is a diagonal matrix with positive diagonal elements.  Moreover, $\mK^{-1} = \mU \mD^{-1} \mU^T$, where $\mU := \mL^{-T}$ is an upper unitriangular matrix. As the sample size grows, these matrices can be constructed recursively.  Expressing the sample size $n$ in the notation, we claim that 
\begin{subequations} \label{eq:cholrecurse}
\begin{gather}
\label{eq:cholrecurseA}
\mK_1  = \mD_1 = K(\bx_1,\bx_1), \quad \mL_1 = \mU_1 = 1,  \quad \\
\intertext{and for $n = 1, 2, \ldots$ we have recursively}
\label{eq:cholrecurseB}\mK_{n+1} = \begin{pmatrix}
\mK_n & K(\mX_n, \bx_{n+1}) \\
K(\bx_{n+1}, \mX_n) & K(\bx_{n+1}, \bx_{n+1})
\end{pmatrix},
\quad 
\mD_{n+1} = \begin{pmatrix}
\mD_n & \bzero_n \\
\bzero_n^T & d_{n+1}
\end{pmatrix},
\\
\label{eq:cholrecurseC}
\mL_{n+1} = \begin{pmatrix}
\mL_n & \bzero_n \\
\bL_{n+1}^T  & 1
\end{pmatrix},
\quad 
\mU_{n+1} = \begin{pmatrix}
\mU_n & \bU_{n+1} \\
\bzero_n^T  & 1
\end{pmatrix}, \\
\intertext{where}
\label{eq:cholrecurseD}
\vell_{n+1} = \mU_n^T K(\mX_n, \bx_{n+1}), \qquad
\bL_{n+1} = \mD_n^{-1} \vell_{n+1}, 
\\
\label{eq:cholrecurseE}
d_{n+1} = K(\bx_{n+1}, \bx_{n+1}) - \vell^T_{n+1} \bL_{n+1}, \qquad
\bU_{n+1} = - \mU_n \bL_{n+1}.
\end{gather}
\end{subequations}

The proof is by induction.  Certain properties follow automatically: the $n = 1$ case,  the recursive form for $\mK_{n+1}$, and the facts that $\mL_n$ is lower unitriangular, the $\mU_n$ is upper unitriangular, and $\mD_n$ is diagonal.  What need to be established are that $\mU_{n+1}= \mL^{-T}_{n+1}$ and that $\mK_{n+1} =  \mL_{n+1} \mD_{n+1} \mL^T_{n+1}$.   Given that  the formulas for $\mD_n$, $\mL_n$, and $\mU_n$ are correct, it follows that 
\begin{align*}
 \mU_{n+1} \mL_{n+1}^T & = 
 \begin{pmatrix}
 \mU_n & \bU_{n+1} \\
 \bzero_n^T  & 1
 \end{pmatrix}
 \begin{pmatrix}
 \mL_n^T & \bL_{n+1} \\
 \bzero_n^T  & 1
 \end{pmatrix}  = 
 \begin{pmatrix}
 \mU_n\mL_n^T & \mU_n\bL_{n+1} + \bU_{n+1}\\
 \bzero_n^T  & 1
 \end{pmatrix} 
 = 
 \begin{pmatrix}
 \mI_n & \bzero_n\\
 \bzero_n^T  & 1
 \end{pmatrix} = \mI_{n+1},\\
 \mL_{n+1} \mD_{n+1} \mL^T_{n+1} &=
 \begin{pmatrix}
\mL_n & \bzero_n \\
\bL_{n+1}^T  & 1
\end{pmatrix}
\begin{pmatrix}
\mD_n & \bzero_n \\
\bzero_n^T & d_{n+1}
\end{pmatrix}
 \begin{pmatrix}
\mL_n^T & \bL_{n+1}  \\
\bzero_n^T  & 1
\end{pmatrix}
 =  \begin{pmatrix}
\mL_n \mD_n \mL_n^T & \mL_n \mD_n \bL_{n+1}  \\
\bL_{n+1}^T \mD_n \mL_n^T   & \bL_{n+1}^T \mD_n \bL_{n+1} + d_{n+1}
\end{pmatrix}\\
& =  \begin{pmatrix}
\mK_n  & \mL_n \mD_n \mD_n^{-1} \vell_{n+1}  \\
\vell_{n+1}^T \mD_n^{-1} \mD_n \mL_n^T   & \vell_{n+1}^T \mD_n^{-1} \mD_n \bL_{n+1}  +  K(\bx_{n+1}, \bx_{n+1}) - \vell^T_{n+1} \bL_{n+1}
\end{pmatrix}\\
& =  \begin{pmatrix}
\mK_n  & \mL_n \mU_n^T K(\mX_n, \bx_{n+1}) \\
 K(\bx_{n+1}, \mX_{n}) \mU_n \mL_n^T   &   K(\bx_{n+1}, \bx_{n+1})
\end{pmatrix} =  \begin{pmatrix}
\mK_n  &  K(\mX_n, \bx_{n+1}) \\
K(\bx_{n+1}, \mX_{n})    &   K(\bx_{n+1}, \bx_{n+1})
\end{pmatrix} = \mK_{n+1}.
\end{align*}
This completes the proof.

Each recursive step in  \eqref{eq:cholrecurse} requires $\Order(n^2)$ operations.  Thus, computing all matrices up to case $n$ requires  $\Order(n^3)$ operations.
\end{document}
